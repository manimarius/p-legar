\documentclass[12pt]{article}
\usepackage[icelandic,english]{babel}
\usepackage{a4,t1enc,html,url}
\usepackage{ucs}
\usepackage{times}
\usepackage[utf8x]{inputenc}
%\pagestyle{empty}
\usepackage{graphicx,latexsym}
\parindent 0pt
%
\usepackage{amsmath} 
\usepackage{graphicx,latexsym} 
\usepackage{amssymb}
\author{Máni Maríus Viðarsson og Jón Blöndal}

\usepackage{amsthm}

\theoremstyle{definition}
\newtheorem{skilgr}{Skilgreining}
\newtheorem{daemi}{Dæmi}
\newtheorem{aefing}{Æfing}

\theoremstyle{plain}
\newtheorem{setn}{Setning}
\newtheorem*{fylgisetn}{Fylgisetning}
\newtheorem*{hjalparsetn}{Hjálparsetning}


%----------------------------------------------------------


\title{p-Legar tölur}
\begin{document}
\maketitle
\section*{Inngangur}
%Hér setjum við sýnidæmi til að vekja áhuga.
Í lok þessara tveggja fyrirlestra mun ykkur vonandi ekki brenna í augun á að sjá eftirfarandi jöfnu:
\begin{equation*}
-1 = 4 + 4*5+4*5^2+4*5^3 \ldots
\end{equation*}
eða 
\begin{equation*}
\sqrt{7} = 1 + 1*3 + 1*3^2+0*3^3+2*3^4 \ldots
\end{equation*}

%Hér kemur fyrsta skilgreining
Festum prímtölu p og skoðum formlegu veldaröðina
\begin{align*}
a = a_0 + a_1 p + a_2 p^2 + ...,
\end{align*}
þar sem $0 \leq a_i \leq p-1$. Þessar veldaraðir er hægt að leggja saman, draga frá og margfalda.
%Hér byrjar svo seinni fyrirlestur


\end{document}
